%Created with csv2Latex.pl
%Pete Mills 2017


\documentclass[12pt,oneside]{memoir}

\usepackage{palatino}
\usepackage[letterpaper,left=1.0in,right=1.0in,bindingoffset=0.0in]{geometry}
\usepackage[nonumber]{cuisine}
\usepackage{xcolor}
\usepackage{float}
\usepackage{graphicx}
\graphicspath{ {images/} }

\setcounter{tocdepth}{2}
%\setcounter{secnumdepth}{4}
\newcommand\invisiblesection[1]{%
\refstepcounter{section}%
\addcontentsline{toc}{section}{\protect\numberline{\thesection}#1}%
\sectionmark{#1}}


%-%-%-%-%-%-%-%-%-%-%-%-%-%-%-%-%_BEGIN TITLE_%-%-%-%-%-%-%-%-%-%-%-%-%-%-%-%-%
\makeatletter

\def\maketitle{%
	\null
	\thispagestyle{empty}
	\vfill
	\begin{center}\leavevmode
		\normalfont
		{\LARGE\raggedleft \@author\par}
		\hrulefill\par
		{\huge\raggedright \@title\par}
		\vskip 1cm
		%{\Large \@date\par}%
	\end{center}%
	\vfill
	\null
	\cleardoublepage
}

\makeatother

\author{Mills Family}
\title{Cookbook}
%-%-%-%-%-%-%-%-%-%-%-%-%-%-%-%-%_END TITLE_%-%-%-%-%-%-%-%-%-%-%-%-%-%-%-%-%-%



%-%-%-%-%-%-%-%-%-%-%-%-%-%-%_BEGIN DOCUMENT_%-%-%-%-%-%-%-%-%-%-%-%-%-%-%-%-%-%
\begin{document}

\newcommand{\recipeName}{blank}
\newcommand{\myImageScalar}{0.75}

\let\cleardoublepage\clearpage
\maketitle
\frontmatter
\chapter*{Preface}
Here is a collection of recipes we like.\newpage% the asterisk means that the contents itself isn't put into the ToC
\tableofcontents*
\mainmatter



\chapter{Drinks}


%_______________START RECIPE______________

% Setup recipe name and image filename
\invisiblesection{Peppermint Mocha}
\renewcommand{\recipeName}{Peppermint Mocha}
\def \imageName {./images/none}
% If no image, use a placeholder image instead
\IfFileExists{\imageName}{}{\def \imageName{./images/placeHolder.jpg}} 
% End setup recipe name and image filename

\begin{recipe}{\recipeName}{1 Portions}{10 Min} 
	\ingredient[250]{mL}{Brewed Coffee}
	\ingredient[1]{Tbsp}{Peppermint Syrup}
	\ingredient[1]{Tbsp}{Vegan Coffee Creamer}
	\ingredient[2]{tsp}{Cocoa Powder}
	Add to a pre-warmed cup and stir to combine.
	\ingredient[1]{}{Vegan Whipped Cream}
	\ingredient[1]{}{Peppermint Candy}
	\ingredient[1]{}{Vegan Chocolate Syrup}
	Garnish as desired.
\end{recipe}


\newpage

%________________END RECIPE_______________






\chapter{Snacks \& Appetizers}


%_______________START RECIPE______________

% Setup recipe name and image filename
\invisiblesection{Hummus}
\renewcommand{\recipeName}{Hummus}
\def \imageName {./images/hummus.jpg}
% If no image, use a placeholder image instead
\IfFileExists{\imageName}{}{\def \imageName{./images/placeHolder.jpg}} 
% End setup recipe name and image filename

\begin{recipe}{\recipeName}{4 Portions}{15 Min} 
	\ingredient[1]{Can}{Chickpeas}
	\ingredient[4]{Cloves}{Garlic}
	\ingredient[3]{Tbsp}{Tahini}
	\ingredient[\fr12]{C}{Lemon Juice}
	\ingredient[1\fr12]{Tsp}{Salt}
	\ingredient[2]{Tbsp}{Parsley}
	\ingredient[1]{Tbsp}{Olive Oil}
	Drain and rinse the chickpeas. Add all ingredients to a blender and blend until smooth. Optionally garnish with whole chickpeas, parsley, and/or olive oil.
\end{recipe}

\begin{figure}[H]
	\centering
	\edef\tmp{\noexpand\includegraphics[width=\myImageScalar\textwidth]{\imageName}}\tmp
	\caption{\recipeName}
\end{figure}

\newpage

%________________END RECIPE_______________






%_______________START RECIPE______________

% Setup recipe name and image filename
\invisiblesection{Tomatillo Salsa Fresca}
\renewcommand{\recipeName}{Tomatillo Salsa Fresca}
\def \imageName {./images/none}
% If no image, use a placeholder image instead
\IfFileExists{\imageName}{}{\def \imageName{./images/placeHolder.jpg}} 
% End setup recipe name and image filename

\begin{recipe}{\recipeName}{4 Portions}{20 Minutes} 
	\ingredient[3]{}{Tomatillos}
	\ingredient[\fr12]{}{Orange or Yellow Bell Pepper}
	\ingredient[1]{}{Anaheim Chillie}
	\ingredient[1]{}{Avocado}
	\ingredient[\fr13]{C}{Chopped Cilantro}
	Wash and chop the ingredients and add to a bowl.
	\ingredient[1]{}{Lime Juice}
	\ingredient[1]{Tbsp}{Agave Nectar}
	\ingredient[\fr14]{tsp}{Salt}
	Drizzle ingredients over the salsa. Mix gently to combine.
\end{recipe}


\newpage

%________________END RECIPE_______________






%_______________START RECIPE______________

% Setup recipe name and image filename
\invisiblesection{Beet Muhummara}
\renewcommand{\recipeName}{Beet Muhummara}
\def \imageName {./images/none}
% If no image, use a placeholder image instead
\IfFileExists{\imageName}{}{\def \imageName{./images/placeHolder.jpg}} 
% End setup recipe name and image filename

\begin{recipe}{\recipeName}{8 Portions}{20 Minutes} 
\freeform Beet Muhummara is a tasty dip with a consistency similar to hummus.	\ingredient[]{4}{Beets - Raw, peeled, and quartered.}
	\ingredient[1\fr13]{C}{Walnuts}
	\ingredient[\fr12]{C}{Panko Bread Crumbs}
	\ingredient[]{3}{Cloves of Garlic}
	\ingredient[1\fr12]{Tbsp}{Ground Cumin}
	\ingredient[\fr12]{tsp}{Salt}
	\ingredient[\fr14]{tsp}{Black Pepper}
	\ingredient[\fr18]{tsp}{Red Pepper Flakes}
	\ingredient[2]{Tbsp}{Pomegranate Molasses}
	\ingredient[3]{Tbsp}{Lemon Juice}
	Combine all ingredients into a food processor.
	\ingredient[\fr14]{C}{Olive Oil}
	With the food processor running, slowly add the olive oil.\\Chill the dip and serve with carrot, celery, pita, and olives.
\end{recipe}


\newpage

%________________END RECIPE_______________






%_______________START RECIPE______________

% Setup recipe name and image filename
\invisiblesection{Black Eyed Pea Hummus}
\renewcommand{\recipeName}{Black Eyed Pea Hummus}
\def \imageName {./images/blackEyedPeaHummus.jpg}
% If no image, use a placeholder image instead
\IfFileExists{\imageName}{}{\def \imageName{./images/placeHolder.jpg}} 
% End setup recipe name and image filename

\begin{recipe}{\recipeName}{8 Portions}{15 Min} 
	\ingredient[2]{Cans}{Black Eyed Peas - Drained and rinsed.}
	\ingredient[\fr12]{}{Onion}
	\ingredient[3]{}{Cloves of Garlic}
	\ingredient[\fr14]{C}{Pecans}
	\ingredient[1\fr12]{Tbsp}{Lemon Juice}
	\ingredient[1]{Tbsp}{Hot Sauce}
	\ingredient[1]{tsp}{Smoked Paprika}
	\ingredient[\fr12]{tsp}{Salt}
	Combine all ingredients in a food processor until smooth.
\end{recipe}

\begin{figure}[H]
	\centering
	\edef\tmp{\noexpand\includegraphics[width=\myImageScalar\textwidth]{\imageName}}\tmp
	\caption{\recipeName}
\end{figure}

\newpage

%________________END RECIPE_______________






%_______________START RECIPE______________

% Setup recipe name and image filename
\invisiblesection{Deviled Tomatoes}
\renewcommand{\recipeName}{Deviled Tomatoes}
\def \imageName {./images/none}
% If no image, use a placeholder image instead
\IfFileExists{\imageName}{}{\def \imageName{./images/placeHolder.jpg}} 
% End setup recipe name and image filename

\begin{recipe}{\recipeName}{8 Portions}{20 Minutes} 
	\ingredient[1]{Can}{Garbanzo Beans - Rinsed, Drained.}
	\ingredient[\fr13]{C}{Vegan Mayo}
	\ingredient[2]{Tbsp}{Nutritional Yeast}
	\ingredient[1]{Tbsp}{Lemon Juice}
	\ingredient[2]{tsp}{Yellow Mustard}
	\ingredient[1]{tsp}{Curry Powder}
	\ingredient[\fr18]{tsp}{Black Pepper}
	\ingredient[\fr18]{tsp}{Cumin}
	\ingredient[\fr18]{tsp}{Salt}
	Mash all ingredients by hand
	\ingredient[]{8 to 10}{Roma Tomatoes}
	Cut each tomato in half and scoop out the insides. Fill the hollow tomato halves with the mixture from step 1.
\end{recipe}


\newpage

%________________END RECIPE_______________






\chapter{Soups}


%_______________START RECIPE______________

% Setup recipe name and image filename
\invisiblesection{Garlic Soup}
\renewcommand{\recipeName}{Garlic Soup}
\def \imageName {./images/none}
% If no image, use a placeholder image instead
\IfFileExists{\imageName}{}{\def \imageName{./images/placeHolder.jpg}} 
% End setup recipe name and image filename

\begin{recipe}{\recipeName}{6 Portions}{20 Min} 
	\ingredient[1]{L}{Water}
	\ingredient[1]{Head}{Garlic}
	\ingredient[1]{tsp}{Salt}
	\ingredient[\fr14]{tsp}{Pepper}
	\ingredient[2]{}{Cloves}
	\ingredient[\fr14]{tsp}{Sage}
	\ingredient[\fr14]{tsp}{Thyme}
	\ingredient[\fr12]{}{Bay Leaf}
	\ingredient[4]{Sprig}{Parsley}
	\ingredient[3]{Tbsp}{Olive Oil}
	Add all ingredients to a sauce pan and simmer for 30 min. Mash the garlic once soft.
\end{recipe}


\newpage

%________________END RECIPE_______________






%_______________START RECIPE______________

% Setup recipe name and image filename
\invisiblesection{Lentil Soup}
\renewcommand{\recipeName}{Lentil Soup}
\def \imageName {./images/none}
% If no image, use a placeholder image instead
\IfFileExists{\imageName}{}{\def \imageName{./images/placeHolder.jpg}} 
% End setup recipe name and image filename

\begin{recipe}{\recipeName}{8 Portions}{1 Hour} 
	\ingredient[2]{}{Large Onions}
	\ingredient[4]{}{Carrots}
	\ingredient[5]{}{Cloves of Garlic}
	Use a large pot and saute in olive oil until onions are translucent.
	\ingredient[4]{C}{Lentils}
	\ingredient[1]{}{Bay Leaf}
	\ingredient[1\fr12]{tsp}{Ground Cumin}
	\ingredient[1\fr12]{tsp}{Ground Corriander}
	\ingredient[1\fr12]{tsp}{Salt}
	\ingredient[\fr14]{tsp}{Turmeric}
	\ingredient[2]{}{Vegan Bullion Cubes}
	Wash lentils and add the remaining ingredients to the pot. Cover with water and bring to a boil. Cover pot with a lid and simmer for 30-40 min. Add water as needed.
\end{recipe}


\newpage

%________________END RECIPE_______________






%_______________START RECIPE______________

% Setup recipe name and image filename
\invisiblesection{Rasam}
\renewcommand{\recipeName}{Rasam}
\def \imageName {./images/none}
% If no image, use a placeholder image instead
\IfFileExists{\imageName}{}{\def \imageName{./images/placeHolder.jpg}} 
% End setup recipe name and image filename

\begin{recipe}{\recipeName}{10 Portions}{45 Min} 
\freeform Rasam is a thin soup from Southern India. This recipe makes a fairly thick version. It is nice to serve with warm rice.	\ingredient[3]{}{Dried Red Chillies}
	\ingredient[1\fr12]{Tbsp}{Dal (Lentil)}
	\ingredient[1\fr12]{Tbsp}{Coriander Seeds}
	\ingredient[1]{tsp}{Cumin Seeds}
	\ingredient[1]{tsp}{Black Peppercorns}
	Add ingredients to a dry pan and toast until fragrant. Then, add to a blender to process into a powder.
	\ingredient[]{28-Oz}{Can of Crushed Tomatoes}
	\ingredient[1\fr12]{L}{Water}
	\ingredient[2]{tsp}{Tamarind Paste}
	\ingredient[1]{Sprig}{Curry Leaves}
	\ingredient[2]{Clove}{Garlic}
	\ingredient[1]{tsp}{Mustard Seed}
	\ingredient[1]{tsp}{Ground Cumin}
	\ingredient[1]{tsp}{Ground Black Pepper}
	\ingredient[\fr12]{tsp}{Methi Seeds (Fenugreek)}
	\ingredient[\fr12]{tsp}{Hing / Asafoetida}
	\ingredient[2]{}{Dried red chillies - broken.}
	\ingredient[1]{tsp}{Mustard Seed}
	Add all ingredients, including the spice powder mix from the previous step, to a sauce pan and simmer for 30 min.
\end{recipe}


\newpage

%________________END RECIPE_______________






%_______________START RECIPE______________

% Setup recipe name and image filename
\invisiblesection{Sweet and Sour Soup}
\renewcommand{\recipeName}{Sweet and Sour Soup}
\def \imageName {./images/SweetAndSourSoup.jpg}
% If no image, use a placeholder image instead
\IfFileExists{\imageName}{}{\def \imageName{./images/placeHolder.jpg}} 
% End setup recipe name and image filename

\begin{recipe}{\recipeName}{4 Portions}{30 Min} 
	\ingredient[1]{Tbsp}{Olive Oil}
	\ingredient[1]{Tbsp}{Grated Ginger}
	Saute in a pot until soft.
	\ingredient[6]{C}{Vegetable Broth}
	\ingredient[\fr12]{Tbsp}{Soy Sauce}
	\ingredient[2]{Tbsp}{Rice Vinegar}
	\ingredient[1]{Tbsp}{Garlic Chili Sauce}
	Add to the pot and bring to a simmer.
	\ingredient[1]{}{Block of Tofu}
	Cut into cubes and add to the pot. Simmer until heated thru.
	\ingredient[\fr14]{}{Red Cabbage}
	\ingredient[4]{}{Green Onions}
	\ingredient[3]{}{Carrots}
	\ingredient[8]{}{Button Mushrooms}
	Slice or julienne vegetables. Divide among serving bowls. Ladle broth and tofu on top.
\end{recipe}

\begin{figure}[H]
	\centering
	\edef\tmp{\noexpand\includegraphics[width=\myImageScalar\textwidth]{\imageName}}\tmp
	\caption{\recipeName}
\end{figure}

\newpage

%________________END RECIPE_______________






%_______________START RECIPE______________

% Setup recipe name and image filename
\invisiblesection{West African Peanut Soup}
\renewcommand{\recipeName}{West African Peanut Soup}
\def \imageName {./images/none}
% If no image, use a placeholder image instead
\IfFileExists{\imageName}{}{\def \imageName{./images/placeHolder.jpg}} 
% End setup recipe name and image filename

\begin{recipe}{\recipeName}{6 Portions}{30 Min} 
	\ingredient[1]{}{Large Yellow Onion}
	\ingredient[\fr12]{Tbsp}{Olive oil}
	Saute until in a pot until translucent.
	\ingredient[2]{}{Carrots}
	\ingredient[\fr14]{Tbsp}{Grated Ginger}
	\ingredient[\fr12]{tsp}{Salt}
	\ingredient[\fr14]{tsp}{Cayenne}
	Peel and chop the carrots. Add the carrots and spices to the pot and continue to saute for 5 minutes longer.
	\ingredient[1]{}{Sweet Potato}
	\ingredient[2\fr12]{C}{Vegetable Stock}
	Peel and chop the sweet potato. Add to the pot with stock and simmer until the vegetables are cooked thru.
	\ingredient[\fr34]{C}{Peanut Butter}
	\ingredient[\fr34]{C}{Vegetable/Tomato Juice}
	Remove from the heat and add the peanut butter and Juice. Blend until smooth.
\end{recipe}


\newpage

%________________END RECIPE_______________






\chapter{Tempeh, Seitan, etc}


%_______________START RECIPE______________

% Setup recipe name and image filename
\invisiblesection{Basic Seitan}
\renewcommand{\recipeName}{Basic Seitan}
\def \imageName {./images/basicSeitan.jpg}
% If no image, use a placeholder image instead
\IfFileExists{\imageName}{}{\def \imageName{./images/placeHolder.jpg}} 
% End setup recipe name and image filename

\begin{recipe}{\recipeName}{10 Portions}{1.5 Hour} 
	\ingredient[1]{Tbsp}{Olive Oil}
	\ingredient[1]{Large}{Onion}
	\ingredient[2]{Cloves}{Garlic}
	Sautee.
	\ingredient[\fr13]{tsp}{Salt}
	\ingredient[1]{tsp}{Paprika}
	\ingredient[2]{Tbsp}{Tomato Paste}
	\ingredient[1]{Tbsp}{Soy Sauce}
	\ingredient[1]{C}{Bullion Broth}
	Combine with ingredients from step 1 in a blender and puree.
	\ingredient[1\fr12]{C}{Vital Wheat Gluten}
	\ingredient[\fr14]{C}{Chickpea Flour}
	\ingredient[2]{Tbsp}{Nutritional Yeast}
	Combine with pureed mixture from step 2 to form a dough. Fill dough into a covered dish for steaming. Steam dish with seitan dough for 1 hour.
\end{recipe}

\begin{figure}[H]
	\centering
	\edef\tmp{\noexpand\includegraphics[width=\myImageScalar\textwidth]{\imageName}}\tmp
	\caption{\recipeName}
\end{figure}

\newpage

%________________END RECIPE_______________






%_______________START RECIPE______________

% Setup recipe name and image filename
\invisiblesection{Greek Gyro Seitan}
\renewcommand{\recipeName}{Greek Gyro Seitan}
\begin{recipe}{\recipeName}{10 Portions}{1.5 Hour} 
	\ingredient[1]{Tbsp}{Olive Oil}
	\ingredient[1]{Large}{Onion}
	\ingredient[2]{Cloves}{Garlic}
	Sautee.
	\ingredient[\fr13]{tsp}{Salt}
	\ingredient[1]{tsp}{Marjoram}
	\ingredient[1]{tsp}{Ground Rosemary}
	\ingredient[2]{Tbsp}{Tomato Paste}
	\ingredient[1]{Tbsp}{Soy Sauce}
	\ingredient[1]{C}{Bullion Broth}
	Combine with ingredients from step 1 in a blender and puree.
	\ingredient[1\fr12]{C}{Vital Wheat Gluten}
	\ingredient[\fr14]{C}{Chickpea Flour}
	\ingredient[2]{Tbsp}{Nutritional Yeast}
	Combine with pureed mixture from step 2 to form a dough. Fill dough into a covered dish for steaming. Steam dish with seitan dough for 1 hour.
\end{recipe}

\begin{figure}[H]
	\centering
	\edef\tmp{\noexpand\includegraphics[width=\myImageScalar\textwidth]{\imageName}}\tmp
	\caption{\recipeName}
\end{figure}

\newpage

%________________END RECIPE_______________






\chapter{Side Dishes}


\chapter{Main Course}


%_______________START RECIPE______________

% Setup recipe name and image filename
\invisiblesection{Pizza Dough}
\renewcommand{\recipeName}{Pizza Dough}
\def \imageName {./images/pizza.jpg}
% If no image, use a placeholder image instead
\IfFileExists{\imageName}{}{\def \imageName{./images/placeHolder.jpg}} 
% End setup recipe name and image filename

\begin{recipe}{\recipeName}{2 Portions}{1 Hour} 
	Preheat the oven to 450 F.
	\ingredient[1]{C}{Warm Water}
	\ingredient[2]{tsp}{Yeast}
	\ingredient[1]{tsp}{Sugar}
	Combine these ingredients and set aside for the yeast to bloom and become frothy. \textasciitilde 10 min.
	\ingredient[360]{g}{AP Flour}
	\ingredient[\fr12]{tsp}{Salt}
	\ingredient[2]{Tbsp}{Olive Oil}
	Combine these ingredients in a bowl or food processor. Add the bloomed yeast mixture from the previous step and combine. \\\\Knead dough until smooth and cover in a warm place to rise. \textasciitilde 30 min.\\\\Roll out the dough, add toppings, and bake for around 10 min. The type and quantity of toppings will affect the bake time.
\end{recipe}

\begin{figure}[H]
	\centering
	\edef\tmp{\noexpand\includegraphics[width=\myImageScalar\textwidth]{\imageName}}\tmp
	\caption{\recipeName}
\end{figure}

\newpage

%________________END RECIPE_______________






\chapter{Deserts}


%_______________START RECIPE______________

% Setup recipe name and image filename
\invisiblesection{Apple Pie}
\renewcommand{\recipeName}{Apple Pie}
\def \imageName {./images/applePie.jpg}
% If no image, use a placeholder image instead
\IfFileExists{\imageName}{}{\def \imageName{./images/placeHolder.jpg}} 
% End setup recipe name and image filename

\begin{recipe}{\recipeName}{1 Pie}{2 Hours} 
	Preheat the oven to 350 F.
	\ingredient[240]{g}{Sifted AP Flour}
	\ingredient[\fr12]{tsp}{Salt}
	\ingredient[\fr12]{tsp}{Sugar}
	Mix ingredients to combine.
	\ingredient[205]{g}{Vegetable shortening}
	Cut in shortening into dry ingredient mixture either by fork or with food processor.
	\ingredient[Up to 6]{Tbsp}{Cold Water}
	Add water 1 Tbsp. at a time until a workable dough has been formed. Refrigerate until needed.
	\ingredient[6-8]{}{Apples}
	Peel then slice apples. Keep slices apples submerged in a water / lemon juice solution to prevent oxidation.
	Roll out /fr12 of the dough and line a pie tin. Drain slightly and add the sliced apples.
	\ingredient[Up to 1]{C}{Sugar}
	\ingredient[\fr14]{tsp}{Salt}
	\ingredient[\fr12]{tsp}{Cinnamon}
	\ingredient[2]{Tbsp}{Flour}
	Combine and sprinkle over the apples in the pie.
	\ingredient[2]{Tbsp}{Vegan Butter}
	Dot pieces of butter onto of apples.
	Roll out remaining dough to form a pie top. Crimp edges and cut off excess dough. Slice steam holes in the dough, and decorate if desired. Bake for 45 min or until pie filling has been bubbling for a minimum of 5 minutes and crust is browned. If the crust is cooking too quickly, cover with aluminum foil.
\end{recipe}

\begin{figure}[H]
	\centering
	\edef\tmp{\noexpand\includegraphics[width=\myImageScalar\textwidth]{\imageName}}\tmp
	\caption{\recipeName}
\end{figure}

\newpage

%________________END RECIPE_______________






\chapter{Staples}


%_______________START RECIPE______________

% Setup recipe name and image filename
\invisiblesection{Almond Milk}
\renewcommand{\recipeName}{Almond Milk}
\begin{recipe}{\recipeName}{4 Cups}{5 Minutes} 
	\ingredient[4]{C}{Cold Water}
	\ingredient[\fr2/3]{C}{Almonds}
	Puree in a blender. Optionally add a sweetener such as a handful of dates, a tablespoon of agave nectar or maple syrup, and/or a teaspoon of vanilla extract.
\end{recipe}

\begin{figure}[H]
	\centering
	\edef\tmp{\noexpand\includegraphics[width=\myImageScalar\textwidth]{\imageName}}\tmp
	\caption{\recipeName}
\end{figure}

\newpage

%________________END RECIPE_______________






%_______________START RECIPE______________

% Setup recipe name and image filename
\invisiblesection{Cashew Milk}
\renewcommand{\recipeName}{Cashew Milk}
\begin{recipe}{\recipeName}{4 Cups}{5 Minutes} 
	\ingredient[4]{C}{Cold Water}
	\ingredient[\fr12]{C}{Cashews}
	Puree in a blender. Optionally add a sweetener such as a handful of dates, a tablespoon of agave nectar or maple syrup, and/or a teaspoon of vanilla extract.
\end{recipe}

\begin{figure}[H]
	\centering
	\edef\tmp{\noexpand\includegraphics[width=\myImageScalar\textwidth]{\imageName}}\tmp
	\caption{\recipeName}
\end{figure}

\newpage

%________________END RECIPE_______________






%_______________START RECIPE______________

% Setup recipe name and image filename
\invisiblesection{Oat Milk}
\renewcommand{\recipeName}{Oat Milk}
\begin{recipe}{\recipeName}{4 Cups}{5 Minutes} 
	\ingredient[4]{C}{Cold Water}
	\ingredient[1]{C}{Oats}
	Puree in a blender. Optionally add a sweetener such as a handful of dates, a tablespoon of agave nectar or maple syrup, and/or a teaspoon of vanilla extract.
\end{recipe}

\begin{figure}[H]
	\centering
	\edef\tmp{\noexpand\includegraphics[width=\myImageScalar\textwidth]{\imageName}}\tmp
	\caption{\recipeName}
\end{figure}

\newpage

%________________END RECIPE_______________






\chapter{Salads}


%_______________START RECIPE______________

% Setup recipe name and image filename
\invisiblesection{Black Rice Salad}
\renewcommand{\recipeName}{Black Rice Salad}
\def \imageName {./images/none}
% If no image, use a placeholder image instead
\IfFileExists{\imageName}{}{\def \imageName{./images/placeHolder.jpg}} 
% End setup recipe name and image filename

\begin{recipe}{\recipeName}{4 Portions}{40 Minutes} 
	\ingredient[3]{C}{Cooked Black Rice}
	\ingredient[\fr14]{C}{Lime Juice}
	\ingredient[1\fr12]{Tbsp}{Olive Oil}
	\ingredient[1]{tsp}{Agave Nectar}
	\ingredient[\fr14]{tsp}{Salt}
	\ingredient[1]{}{Minced Garlic}
	\ingredient[2]{Tbsp}{Chopped Cilantro}
	Make vinaigrette and pour over cooked black rice. Cover and chill.
	\ingredient[1]{}{Diced Mango}
	\ingredient[1]{}{Diced Avocado}
	\ingredient[\fr12]{C}{Chopped Red Onion}
	\ingredient[\fr12]{C}{Chopped Cilantro}
	Add to the salad, cover and chill again before serving.
\end{recipe}


\newpage

%________________END RECIPE_______________






%_______________START RECIPE______________

% Setup recipe name and image filename
\invisiblesection{Chickpea Salad}
\renewcommand{\recipeName}{Chickpea Salad}
\def \imageName {./images/none}
% If no image, use a placeholder image instead
\IfFileExists{\imageName}{}{\def \imageName{./images/placeHolder.jpg}} 
% End setup recipe name and image filename

\begin{recipe}{\recipeName}{4 Portions}{15 Minutes} 
	\ingredient[1]{}{Can Chickpeas}
	\ingredient[2]{Tbsp}{Sweet Pickle Relish}
	\ingredient[\fr12]{tsp}{Creole Seasoning}
	\ingredient[2]{Tbsp}{Vegan Mayo}
	Drain and rinse the chickpeas. Add all to a bowl and mix.
\end{recipe}


\newpage

%________________END RECIPE_______________






%_______________START RECIPE______________

% Setup recipe name and image filename
\invisiblesection{Cranberry Salad}
\renewcommand{\recipeName}{Cranberry Salad}
\def \imageName {./images/none}
% If no image, use a placeholder image instead
\IfFileExists{\imageName}{}{\def \imageName{./images/placeHolder.jpg}} 
% End setup recipe name and image filename

\begin{recipe}{\recipeName}{4 Portions}{20 Minutes} 
	\ingredient[\fr12]{C}{Sugar}
	\ingredient[1]{}{Red Vegan Jello}
	\ingredient[2]{}{Oranges}
	\ingredient[1]{Lb}{Fresh Cranberries}
	Peel one of the oranges. Add all to a food processor and chop. Transfer to a bowl.
	\ingredient[1]{C}{Celery}
	\ingredient[1]{C}{Apples}
	\ingredient[1]{C}{Walnuts}
	Chop in a food processor, then add to the bowl. Stir to combine, and chill for 3 hours.
\end{recipe}


\newpage

%________________END RECIPE_______________






%_______________START RECIPE______________

% Setup recipe name and image filename
\invisiblesection{Vegan Tuna Salad}
\renewcommand{\recipeName}{Vegan Tuna Salad}
\def \imageName {./images/none}
% If no image, use a placeholder image instead
\IfFileExists{\imageName}{}{\def \imageName{./images/placeHolder.jpg}} 
% End setup recipe name and image filename

\begin{recipe}{\recipeName}{4 Portions}{15 Minutes} 
	\ingredient[1]{}{Can Chickpeas}
	\ingredient[2]{}{Chopped Celery Stalks}
	\ingredient[\fr13]{C}{Chopped Dill Pickle}
	\ingredient[2]{Tbsp}{Vegan Mayo}
	\ingredient[1]{Tsp}{White Vinegar}
	\ingredient[1]{Tbsp}{Chopped Organic Kelp}
	\ingredient[To Taste]{}{Salt \& Pepper}
	Drain and rinse the chickpeas. Add all to a bowl and mix.
\end{recipe}

\begin{flushleft}
	Note: This is even better the next day.
\end{flushleft}


\newpage

%________________END RECIPE_______________




\end{document}
%-%-%-%-%-%-%-%-%-%-%-%-%-%-%-%_END DOCUMENT_%-%-%-%-%-%-%-%-%-%-%-%-%-%-%-%-%-%
